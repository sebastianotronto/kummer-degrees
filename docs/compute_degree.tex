\documentclass[10pt,a4paper]{article}
\usepackage[utf8]{inputenc}
\usepackage{amsmath}
\usepackage{amsthm}
\usepackage[all]{xy}
\usepackage{amsfonts}
\usepackage{color}
\usepackage{amssymb}
\usepackage{float}
\usepackage[a4paper, top=3cm, bottom=3cm, left=2.5cm, right=2.5cm]{geometry}

\DeclareMathOperator{\alg}{alg}
\DeclareMathOperator{\obj}{Obj}
\DeclareMathOperator{\Hom}{Hom}
\DeclareMathOperator{\End}{End}
\DeclareMathOperator{\hol}{Hol}
\DeclareMathOperator{\aut}{Aut}
\DeclareMathOperator{\gal}{Gal}
\DeclareMathOperator{\id}{id}
\DeclareMathOperator{\res}{res}
\DeclareMathOperator{\im}{Im}
\DeclareMathOperator{\Id}{Id}
\DeclareMathOperator{\fib}{Fib}
\DeclareMathOperator{\spec}{Spec}
\DeclareMathOperator{\proj}{Proj}
\DeclareMathOperator{\trdeg}{trdeg}
\DeclareMathOperator{\car}{char}
\DeclareMathOperator{\Frac}{Frac}
\DeclareMathOperator{\reduced}{red}
\DeclareMathOperator{\real}{Re}
\DeclareMathOperator{\imag}{Im}
\DeclareMathOperator{\vol}{vol}
\DeclareMathOperator{\den}{den}
\DeclareMathOperator{\rank}{rank}
\DeclareMathOperator{\lcm}{lcm}
\DeclareMathOperator{\rad}{rad}
\DeclareMathOperator{\ord}{ord}
\DeclareMathOperator{\Br}{Br}
\DeclareMathOperator{\inv}{inv}
\DeclareMathOperator{\Nm}{Nm}
\DeclareMathOperator{\Tr}{Tr}
\DeclareMathOperator{\an}{an}
\DeclareMathOperator{\op}{op}
\DeclareMathOperator{\sep}{sep}
\DeclareMathOperator{\unr}{unr}
\DeclareMathOperator{\et}{\acute et}
\DeclareMathOperator{\ev}{ev}
\DeclareMathOperator{\gl}{GL}
\DeclareMathOperator{\SL}{SL}
\DeclareMathOperator{\mat}{Mat}
\DeclareMathOperator{\ab}{ab}
\DeclareMathOperator{\tors}{tors}
\DeclareMathOperator{\ed}{ed}

\newcommand{\grp}{\textsc{Grp}}
\newcommand{\set}{\textsc{Set}}
\newcommand{\x}{\mathbf{x}}
\newcommand{\naturalto}{\overset{.}{\to}}
\newcommand{\qbar}{\overline{\mathbb{Q}}}
\newcommand{\zbar}{\overline{\mathbb{Z}}}

\newcommand{\pro}{\mathbb{P}}
\newcommand{\aff}{\mathbb{A}}
\newcommand{\quat}{\mathbb{H}}
\newcommand{\rea}{\mathbb{R}}
\newcommand{\kiu}{\mathbb{Q}}
\newcommand{\F}{\mathbb{F}}
\newcommand{\zee}{\mathbb{Z}}
\newcommand{\ow}{\mathcal{O}}
\newcommand{\mcx}{\mathcal{X}}
\newcommand{\mcy}{\mathcal{Y}}
\newcommand{\mcs}{\mathcal{S}}
\newcommand{\mca}{\mathcal{A}}
\newcommand{\mcb}{\mathcal{B}}
\newcommand{\mcf}{\mathcal{F}}
\newcommand{\mcg}{\mathcal{G}}
\newcommand{\mct}{\mathcal{T}}
\newcommand{\mcq}{\mathcal{Q}}
\newcommand{\mcr}{\mathcal{R}}
\newcommand{\adl}{\mathbf{A}}
\newcommand{\mbk}{\mathbf{k}}
\newcommand{\m}{\mathfrak{m}}
\newcommand{\p}{\mathfrak{p}}

\newcommand{\kbar}{\overline{K}}

\newtheorem{lemma}{Lemma}
\newtheorem{proposition}[lemma]{Proposition}
\newtheorem{conjecture}[lemma]{Conjecture}
\newtheorem{corollary}[lemma]{Corollary}
\newtheorem{definition}[lemma]{Definition}
\newtheorem{theorem}[lemma]{Theorem}
\newtheorem{cond-thm}[lemma]{Conditional Theorem}
\theoremstyle{definition}
\newtheorem{remark}[lemma]{Remark}

\author{Sebastiano Tronto}


\begin{document}

\begin{lemma}
\label{lemma_zero}
Let $H\leq \mathbb{Q}^\times$ be a finitely generated subgroup. Assume that $H$ does not contain minus a square of $\mathbb{Q}^\times$ or that $m=1$. Then we have
\begin{align*}
\left[\mathbb{Q}_{2^m}\left(\sqrt{H}\right):\mathbb{Q}_{2^m}\right]=\begin{cases}
\#\overline H/2 & \text{ if }m\geq 3\text{ and }\exists b\in H\text{ with }b\equiv\pm2\pmod{\mathbb{Q}^{\times 2}},\\
\#\overline H&\text{ otherwise}.
\end{cases}
\end{align*}
where $\overline{H}$ is the image of $H\cdot \mathbb{Q}^{\times 2}$ in $\mathbb{Q}^\times/\mathbb{Q}^{\times 2}$.
\begin{proof}
Clearly we may assume that $H$ is generated by suqarefree integers $\{g_1,\dots, g_r\}$, where $r=\#\overline H$. In fact, we have that $\mathbb{Q}_{2^m}(\sqrt{H})=\mathbb{Q}_{2^m}(\sqrt{H'})$ for any $H'$ such that $(H\cdot \mathbb{Q}^{\times 2})/\mathbb{Q}^{\times 2}=(H'\cdot \mathbb{Q}^{\times 2})/\mathbb{Q}^{\times 2}$. Recall moreover that by {\color{red}Lemma 13} if there is $\pm2$ times a square in $H$ we can assume that, say, $g_1=\pm 2$.

Assume first that $m\geq 2$, so that $-1\not\in H$ by assumption. In this case we can work over $\mathbb Q_4$ and use Theorem 18 of \cite{DebryPerucca}. We just need to compute the divisibility parameters over $\mathbb{Q}_4$:
\begin{align*}
d_1=\begin{cases}
0&\text{ if }g_1\neq\pm2\\
1&\text{ if }g_1=\pm2
\end{cases},
&&
d_i=0
\quad \text{ for $i=2,\dots, r$},\\
h_1=\begin{cases}
0&\text{ if } 0\leq g_1\neq2\\
1&\text{ if } -2\neq g_1<0\\
2&\text{ if } g_1=\pm 2
\end{cases}, &&
h_i=\begin{cases}
0&\text{ if }g_i>0\\
1&\text{ if }g_i<0
\end{cases}
\quad \text{ for $i=2,\dots, r$}.
\end{align*}
Thus, keeping the notation of the aformentioned Theorem, we get
\begin{align*}
n_1=\min(1,d_1)=\begin{cases}
0&\text{ if }g_1\neq\pm2\\
1&\text{ if }g_1=\pm2
\end{cases},&& n_i=0\quad \text{ for $i=2,\dots, r$}.
\end{align*}
Thus we get
\begin{align*}
v_2\left[\mathbb{Q}_{2^m}(\sqrt{H}):\mathbb Q_{2^m}\right]&=\max(h_1+n_1,\dots, h_r+n_r,m)-m+r-\sum_{i=1}^rn_i=\\
&=\begin{cases}
\max(3,m)-m+r-\sum_{i=1}^rn_i&\text{ if }\pm2\in H\\
r-\sum_{i=1}^rn_i&\text{ if }\pm2\not \in H
\end{cases}\\
&=\begin{cases}
1+r-1&\text{ if }m=2\text{ and }\pm2\in H\\
r-1&\text{ if }m\geq3\text{ and }\pm2\in H\\
r&\text{ if }\pm2\not\in H
\end{cases}
\end{align*}
which is what we want.

Assume now that $m=1$. If $-1\not\in H$, we get the desired result directly from Lemma 19 of \cite{DebryPerucca} applied with $G=H$, using the computations that we did in the previous case. In case $-1\in H$, let $H'$ be any subgroup of $H$ such that $H=H'\oplus\langle-1\rangle$. Notice that we have $\#\overline {H'}=r-1$, so that Lemma 19 with $G=H'$ again gives our result, and the Proposition is proved.
\end{proof}
\end{lemma}

Let $G\leq \mathbb{Q}^\times$ be a finitely generated torsion-free subgroup of rank $r$ and let $M$ and $n$ be integers such that $2^n\,|\,M$. We want to compute the degree
\begin{align}
\label{degree}
\left[\mathbb{Q}_{2^n}\left(G^{1/2^n}\right)\cap \mathbb{Q}_M:\mathbb{Q}_{2^n}\right].
\end{align}

We will use the same notation as that of Remark 17 of Pietro's file.

\section{Case $G\leq \mathbb{Q}_+^\times$}

Assume that $G\leq \mathbb{Q}_+^\times$. In this case, by Remark 17, we have that
\begin{align*}
\mathbb{Q}_{2^n}\left(G^{1/2^n}\right)\cap \mathbb{Q}_M =\mathbb{Q}_{2^n}\left(\sqrt{H}\right).
\end{align*}

Let $\overline{H}$ be the image of $H$ in $\mathbb{Q^\times}/\mathbb{Q}^{\times 2}$. By Remark 17 and Lemma \ref{lemma_zero} above, the degree (\ref{degree}) is given by
\begin{align*}
\left[\mathbb{Q}_{2^n}\left(G^{1/2^n}\right)\cap \mathbb{Q}_M:\mathbb{Q}_{2^n}\right]=
\begin{cases}
\#\overline H/2 & \text{if }n\geq 3\text{ and }2\in H,\\
\#\overline H&\text{ otherwise}.
\end{cases}
\end{align*}

\section{General case}

Let $\mathcal{B}$ be a basis for $G$ and let $\mathcal{B}_i\subseteq \mathcal{B}$ be the subset of basis elements of $2$-divisibility $i$. Call also $L=\max d_i$ the largest $2$-divisiblity parameter. In this way $\mathcal{B}_0,\dots,\mathcal{B}_L$ is a partition of $\mathcal{B}$.

As explained in ({\color{red}ref}) we may assume that there is at most one negative basis element. Since we have dealt with the $G\subseteq \mathbb{Q}_+$ case in the previous section, we assume that such an element exists and that it has $2$-divisibility $d$. We call this element $g_0$.

It is (or will be?) clear ({\color{red}but we should explain it}) that it actually does not matter if we have negative elements of divisibility $0$: that case is treated exactly as the case $G\subseteq \mathbb{Q}_+$. In conclusion, we assume that:
\begin{align*}
\mathcal{B}_1,\dots,\mathcal{B}_{d-1},\mathcal{B}_{d+1},\dots,\mathcal{B}_L\subseteq \mathbb{Q}_+,\\
g_0<0 \text{ and }\mathcal{B}_d\setminus \{g_0\}\subseteq \mathbb{Q}_+,\\
d\geq 1.
\end{align*}

We also let
\begin{align*}
N=\begin{cases}
\max(3,L)&\text{if }d\neq L,\\
\max(3,L+1)&\text{if }d=L.
\end{cases}
\end{align*}

\subsection{General case, $n=1(\leq d)$}
This case can be treated as follows: let $\mathcal{S}'=\mathcal{S}\cup \{-1\}$ and let $H'$ be constructed from $\mathcal{S}'$ in the exact same way as $H$ is constructed from $\mathcal{S}$. Then it's easy to check ({\color{red}it follows from the ``torsion case'' for $G$, it is for sure in some other file}) that $\mathbb{Q}_{2^n}\left(\sqrt{H'}\right)=\mathbb{Q}_{2^{w'}}\left(\sqrt{H}\right)$, where $w'=\min(v_2(M),n+1)$ (as in Remark 17). Then we can again use Lemma \ref{lemma_zero} and conclude that
\begin{align*}
\left[\mathbb{Q}_{2^n}\left(G^{1/2^n}\right)\cap \mathbb{Q}_M:\mathbb{Q}_{2^n}\right]=
\#\overline{H'},
\end{align*}
where $\#\overline{H'}$ is the image of $H'$ in $\mathbb{Q}^\times/\mathbb{Q}^{\times 2}$.

\subsection{General case, $n=2\leq d$}
We consider two cases:
\begin{itemize}
\item If $v_2(M)=2$ we have $\left[\mathbb{Q}_{2^n}\left(G^{1/2^n}\right)\cap \mathbb{Q}_M:\mathbb{Q}_{2^n}\right]=\left[\mathbb{Q}_4\left(\sqrt{H}\right):\mathbb{Q}_4\right]=\#\overline{H}$ by Lemma \ref{lemma_zero}.
\item If $v_2(M)\geq 3$ we have
\begin{align*}
\left[\mathbb{Q}_{2^n}\left(G^{1/2^n}\right)\cap \mathbb{Q}_M:\mathbb{Q}_{2^n}\right]&=\left[\mathbb{Q}_8\left(\sqrt{H}\right):\mathbb{Q}_4\right]=\left[\mathbb{Q}_8\left(\sqrt{H}\right):\mathbb{Q}_8\right]\cdot \left[\mathbb{Q}_8:\mathbb{Q}_4\right]=\\&=2\left[\mathbb{Q}_8\left(\sqrt{H}\right):\mathbb{Q}_8\right],
\end{align*}
which, by Lemma \ref{lemma_zero}, is given by $\#\overline{H}$ if $2 \in H$ and by $2\#\overline{H}$ otherwise.
\end{itemize}

\subsection{General case, $3\leq n\leq d$}
We consider two cases:
\begin{itemize}
\item If $v_2(M)=3$, by lemma \ref{lemma_zero} we have
\begin{align*}
\left[\mathbb{Q}_{2^n}\left(G^{1/2^n}\right)\cap \mathbb{Q}_M:\mathbb{Q}_{2^n}\right]=\left[\mathbb{Q}_8\left(\sqrt{H}\right):\mathbb{Q}_8\right]=\begin{cases}
\#\overline H/2 & \text{ if }\pm 2\in H,\\
\#\overline H&\text{ otherwise}.
\end{cases}
\end{align*}
\item If $v_2(M)\geq 4$ we have
\begin{align*}
\left[\mathbb{Q}_{2^n}\left(G^{1/2^n}\right)\cap \mathbb{Q}_M:\mathbb{Q}_{2^n}\right]&=\left[\mathbb{Q}_{16}\left(\sqrt{H}\right):\mathbb{Q}_8\right]=\left[\mathbb{Q}_{16}\left(\sqrt{H}\right):\mathbb{Q}_{16}\right]\cdot \left[\mathbb{Q}_{16}:\mathbb{Q}_8\right]=\\&=2\left[\mathbb{Q}_{16}\left(\sqrt{H}\right):\mathbb{Q}_{16}\right],
\end{align*}
which, by Lemma \ref{lemma_zero}, is given by $\#\overline{H}$ if $2 \in H$ and by $2\#\overline{H}$ otherwise.
\end{itemize}

\subsection{General case, $n\geq d+2$}
By the corresponding case in Remark 17, we simply have
\begin{align*}
\left[\mathbb{Q}_{2^n}\left(G^{1/2^n}\right)\cap \mathbb{Q}_M:\mathbb{Q}_{2^n}\right]=\begin{cases}
\#\overline {H'}/2 & \text{ if }\pm 2\in H,\\
\#\overline {H'}&\text{ otherwise}.
\end{cases}
\end{align*}
where $H'$ is constructed from $\mathcal{S}'=\mathcal{S}\cup\{B_0\}$ and $\overline{H'}$ is the image of $H'$ in $\mathbb{Q}^\times/\mathbb{Q}^{\times 2}$.

\subsection{General case, $n=d+1$}
We distinguish between some cases.
\begin{itemize}
\item Assume $n=2$ (thus $d=3$) and $v_2(g_0)=2$ (i.e. $2$ divides the square-free part of $B_0$, where $g_0=-B_0^{2^d}$). Then we write the square-free part of $B_0$ as $2s$ for some odd square-free $s\in\mathbb{Z}$. Then letting $\mathcal{S}':=\mathcal{S}\cup \{s\}$ and construct $H'$ from $\mathcal{S}'$ in the usual way. By Remark 17 we have
\begin{align*}
\left[\mathbb{Q}_{2^n}\left(G^{1/2^n}\right)\cap \mathbb{Q}_M:\mathbb{Q}_{2^n}\right]=\left[\mathbb{Q}_{2^n}\left(\sqrt{H'}\right):\mathbb{Q}_{2^n}\right]=\#\overline{H'}.
\end{align*}
But we can be more precise and say that
\begin{align*}
\#\overline{H'}=\begin{cases}
2\#\overline{H}&\text{if }\sqrt{xs}\in\mathbb{Q}_M\text{ for some }x\in\mathcal{S}\text{ and }s\not\in \mathcal{S},\\
\#\overline{H}&\text{otherwise}.
\end{cases}
\end{align*}
%\item Assume $n=2$, $2^{n+1}\nmid M$ and either $v_2(g_0)>2$ or $g_0$ is odd. Then $\left[\mathbb{Q}_{2^n}\left(G^{1/2^n}\right)\cap \mathbb{Q}_M:\mathbb{Q}_{2^n}\right]=\#\overline H$.
%\item Assume $n=2$, $2^{n+1}\,|\,M$ and either $v_2(g_0)>2$ or $g_0$ is odd. ({\color{red}TODO})
\item Assume $n\geq 2$ and $2^{n+1}\nmid M$. Then
\begin{align*}
\left[\mathbb{Q}_{2^n}\left(G^{1/2^n}\right)\cap \mathbb{Q}_M:\mathbb{Q}_{2^n}\right]=\begin{cases}
\#\overline H/2 & \text{ if }\pm 2\in H\text{ and }n\geq 3,\\
\#\overline H&\text{ otherwise}.
\end{cases}
\end{align*}
\item Assume $n\geq 2$ and $2^{n+1}\,|\,M$. Following the notation of Remark 17, we have
\begin{align*}
\mathbb{Q}_{2^n}\left(G^{1/2^n}\right)\cap \mathbb{Q}_M=\mathbb{Q}_{2^n}\left(\sqrt{\langle H, H'\rangle}\right)
\end{align*}
hence
\begin{align*}
\left[\mathbb{Q}_{2^n}\left(G^{1/2^n}\right)\cap \mathbb{Q}_M:\mathbb{Q}_{2^n}\right]&=\left[\mathbb{Q}_{2^n}\left(\sqrt{\langle H, H'\rangle}\right):\mathbb{Q}_{2^n}\right]=\\
&=\left[\mathbb{Q}_{2^n}\left(\sqrt{\langle H, H'\rangle}\right):\mathbb{Q}_{2^n}\left(\sqrt{H}\right)\right]\cdot \left[\mathbb{Q}_{2^n}\left(\sqrt{H}\right):\mathbb{Q}_{2^n}\right].
\end{align*}
We claim that
\begin{align*}
\left[\mathbb{Q}_{2^n}\left(\sqrt{\langle H, H'\rangle}\right):\mathbb{Q}_{2^n}\left(\sqrt{H}\right)\right]=\begin{cases}
1&\text{ if }H'=\emptyset\text{ or }H'=\{2\zeta_4\},\\
2&\text{ otherwise}.
\end{cases}
\end{align*}
To see this, notice that $\sqrt{2\zeta_4}=\zeta_8\sqrt{2}\in\mathbb{Q}_4\subseteq\mathbb{Q}_{2^n}\left(\sqrt{H}\right)$, so the first case is settled. Assume now that there is $x=\zeta_{2^n}b\in H'$ with $x\neq 2\zeta_4$. If $y=\zeta_{2^n}c$ is any other element of $H'$, then we have $\sqrt{x/y}=\sqrt{b/c}$. So if $x,y\in \mathbb{Q}_{2^n}\left(\sqrt{\langle H, H'\rangle}\right)$ we have also $\sqrt{b/c}\in \mathbb{Q}_{2^n}\left(\sqrt{\langle H, H'\rangle}\right)$, which by Kummer theory implies $bc\in H$. But then $y\in \mathbb{Q}_{2^n}\left(\sqrt{H}\right)\left(x\right)$. So we have $\mathbb{Q}_{2^n}\left(\sqrt{\langle H, H'\rangle}\right)=\mathbb{Q}_{2^n}\left(\sqrt{H}\right)(x)$, and the sought degree is $\left[\mathbb{Q}_{2^n}\left(\sqrt{H}\right)(x):\mathbb{Q}_{2^n}\left(\sqrt{H}\right)\right]$, which is in fact $2$ ({\color{red}Do we need to explain this better?}).

We conclude that
\begin{align*}
\left[\mathbb{Q}_{2^n}\left(G^{1/2^n}\right)\cap \mathbb{Q}_M:\mathbb{Q}_{2^n}\right]=\begin{cases}
\#\overline{H}/2&\text{ if } n\geq 3,\,\pm2 \in H\text{ and }H'\subseteq\{2\zeta_4\},\\
\#\overline{H}&\text{ if }(n<3\text{ or }\pm2\not\in H)\text{ and }H'\subseteq \{2\zeta_4\},\\
\#\overline{H}&\text{ if } n\geq 3,\,\pm2 \in H\text{ and }H'\not\subseteq\{2\zeta_4\},\\
2\cdot \#\overline{H}&\text{ if }(n<3\text{ or }\pm2\not\in H)\text{ and }H'\not\subseteq \{2\zeta_4\}.
\end{cases}
\end{align*}
%Let $s$ as in the first subcase of this section and let $\mathcal{C}'$ and $H'$ be as in the last case of Remark 17. We have
%\begin{align*}
%\left[\mathbb{Q}_{2^n}\left(G^{1/2^n}\right)\cap \mathbb{Q}_M:\mathbb{Q}_{2^n}\right]=&\left[\mathbb{Q}_{2^n}\left(\sqrt{\langle H,\zeta_{2^n}H'\rangle}\right):\mathbb{Q}_{2^n}\right]=\\
%=&\left[\mathbb{Q}_{2^n}\left(\sqrt{\langle H,\zeta_{2^n}H'\rangle}\right):\mathbb{Q}_{2^n}\left(\sqrt{H}\right)\right]\cdot \left[\mathbb{Q}_{2^n}\left(\sqrt{H}\right):\mathbb{Q}_{2^n}\right].
%\end{align*}
%Notice that, by construction of $H$ and $H'$, the degree $\left[\mathbb{Q}_{2^n}\left(\sqrt{\langle H,\zeta_{2^n}H'\rangle}\right):\mathbb{Q}_{2^n}\left(\sqrt{H}\right)\right]$ is either $1$
\end{itemize}

\begin{thebibliography}{10} \expandafter\ifx\csname url\endcsname\relax   \def\url#1{\texttt{#1}}\fi \expandafter\ifx\csname urlprefix\endcsname\relax\def\urlprefix{URL }\fi

\bibitem{DebryPerucca}  
\textsc{Debry, C. - Perucca, A.}: \emph{Reductions of algebraic integers}, J. Number Theory, {\bf 167} (2016), 259--283.

\bibitem{PeruccaSgobba}  
\textsc{Perucca, A. - Sgobba, P.}: \emph{Kummer Theory for Number Fields}, preprint.

\end{thebibliography}

\end{document}